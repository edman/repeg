
%%%%%%%%%%%%%%%%%%%%%%%%%%%%%%%%%%%%%%%%%%%%%%
%% RegexLib Documentation
\documentclass[12pt]{article}

% directory for the files to be included
\newcommand{\filesdir}{tex-files}
% name of the library
\newcommand{\libname}{RePeg}
% font style for regular expressions
\newcommand{\markpat}[1] {\texttt{#1}}


\title{\libname}
\author{Edman Paes Anjos \and S�rgio Queiroz de Medeiros}

\begin{document}

% title and initial page
\maketitle
%A pattern matching tool that evaluates regular expressions using Parsing Expression Grammars (PEGs)
A pattern matching tool that translates regular expressions to identical Parsing
Expression Grammars (PEGs). We implement the regular expression's semantics
using PEGs.
, so that the user doe.

%%%%%%%%%%%%%%%%%%%%%%%%%%%%%%%%%%%%%%%%%%%%%%
%% Introduction
\section{The \libname{} Library}
\label{sec:introduction}

\libname{} is a library for pattern matching in the Lua programming language.
It uses most of the traditional and well-known PCRL syntax of regular
expressions, therefore it does not requires the user any new knowledge,
providing a familiar environment. In this text you can find a reference manual
for the library.

%The following table describes the syntax recognized by \libname{} for regular
The table \ref{tab:syntax} describes the syntax recognized by \libname{} for regular
expressions. Here the \markpat{a} or \markpat{b} represent a single character;
\markpat{s} represents a string of characters; \markpat{p} represents a pattern;
\markpat{num} represents a number (\verb=[0-9]+=).

\begin{table}[!h]
	\label{tab:syntax}
	\centering
	\begin{tabular}{|l|l|}
		\hline
		\textbf{Syntax}					&		\textbf{Description} \\
		\hline

		% grouping and capture
		\markpat{(?: p )}				&		grouping \\
		\markpat{( p )}					&		capture \\
		% terminal definitions
		\markpat{.}						&		any character \\
		\markpat{`'}					&		empty string \\
		\markpat{`s'}					&		literal string \\
		\markpat{\$}					&		end of input \\
		\textbackslash\markpat{z}		&		end of line or end of input \\
		\textbackslash\markpat{Z}		&		end of input preceded or not by
			end of line\\
		\markpat{[a-b]}					&		character range \\
		% choice and concatenation
		\markpat{p1 / p2}				&		choice \\
		\markpat{p1 p2}					&		concatenation \\
		% predicates
		\markpat{?= p}					&		and predicate \\
		\markpat{?!~ p}					&		not predicate \\
		% repetitions
		\markpat{p ?}					&		optional match \\
		\markpat{p *}					&		zero or more repetitions \\
		\markpat{p +}					&		one or more repetitions \\
		\markpat{p *+}					&		possessive repetition \\
		\markpat{p *?}					&		lazy repetition \\
		\markpat{p \{ num \}}			&		exactly \markpat{num} repetitions \\
		\markpat{p \{ num , \}}			&		\markpat{num} repetitions or more \\
		\markpat{p \{ num1 , num2 \}}	&		between \markpat{num1} and
			\markpat{num2} repetitions, inclusive \\

		\hline
	\end{tabular}
\end{table}

%%%%%%%%%%%%%%%%%%%%%%%%%%%%%%%%%%%%%%%%%%%%%%
%% Functions
\section{Functions}
\label{sec:functions}

%%% Function match
\subsection{\markpat{\libname{}.match (pattern, subject)}}
\markpat{pattern} $\rightarrow$ a string describing a regular expression
\\ \markpat{subject} $\rightarrow$ the string of characters to be matched
against the \markpat{pattern}

% match description
Matches directly a \markpat{pattern} to a string, returning the portion of the
\markpat{subject} succesfully matched.


%%% Function find
\subsection{\markpat{\libname{}.find (pattern, subject)}}
\markpat{pattern} $\rightarrow$ a string describing a regular expression
\\ \markpat{subject} $\rightarrow$ the string of characters to be matched
against the \markpat{pattern}

% find description
Seeks for the first substring of the \markpat{subject} that can be matched by
the given \markpat{pattern}. If it matches more than one substring, return the
largest.

%%%%%%%%%%%%%%%%%%%%%%%%%%%%%%%%%%%%%%%%%%%%%%
%% Usage Examples
\section{Usage Examples}
\label{sec:examples}

%%% A simple program
\subsection{A simple program}
The following code specifies a running Lua program. In this case, both calls for
\markpat{find} and \markpat{match} yeld the same result and could be used
interchangeably.

\begin{verbatim}
repeg = require 'repeg'

-- find the first number in a string
string = "this string has 29 characters"
print(repeg.find("[0-9]+", string)) --> {28}
print(repeg.match(".*? ([0-9]+)", string)); --> {28}
\end{verbatim}


%%%

\end{document}

